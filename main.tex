% ===========================================================================
% Template LaTeX Doc for Linear Algebra students.
% To compile, click "PDF" on the left. (May have to click more than once)
% Copy/Paste this template for each new HW you create.  
% Go to settings on left to choose which .tex file to compile. 
% To add a new .tex file, just click the little "+" at the very bottom left of the screen
% You can only add new .tex files to your own projects. 
% ===========================================================================

\documentclass{article}

% Packages
\usepackage[utf8]{inputenc}     % Unicode input 
\usepackage{amsmath, amssymb}   % American Math Society Foratting & Symbols
\usepackage{graphicx}           % Allows you to insert graphics
\usepackage{hyperref}           % Allows automatic references & hyperlinking
\usepackage{xfrac}

% Macros (these are shortcuts)
% Read more about them here 
% http://www.math.uh.edu/~torok/math_6298/latex/macros.html
\newcommand{\R}{{\mathbb R}}    % Type $\R$ instead of $\mathbb{R}$ for the real numbers styliezed R
\newcommand{\vt}{\mathbf}       % Type $\vt{...}$ instead of $\mathbf{...}$ to bold a symbol.

\begin{document}

% Title Stuff
\title{NLP: Homework 1: Background Review + Building a classifier}
\author{Raef Youssef}
\date{2024/01/26}
\maketitle                      % This command makes the title

\section{Refreshing the Rows and Columns: Linear Algebra Review}
% Section 1.1
\subsection{Basic Operations}
\begin{itemize}
    \item Givens
    \[
        \alpha = 2, 
        x = \begin{bmatrix} 
            0  \\ 
            1  \\ 
            2  
        \end{bmatrix},  
        y = \begin{bmatrix} 
            3  \\ 
            4  \\ 
            5  
        \end{bmatrix}, 
         z = \begin{bmatrix} 
            1  \\ 
            2  \\ 
            -1  
        \end{bmatrix}, 
        A = \begin{bmatrix} 
            3 & 2 & 2  \\ 
            1 & 3 & 1  \\ 
            1 & 1 & 3  
        \end{bmatrix}
    \]
\end{itemize}
\begin{enumerate}
    \item 
    \[x * y = 0 + 4 + 10 = 14\]
    \item 
    \[x * z = 0 + 2 -2 = 0\]
    \item 
    \[\alpha(x + y) = 
        \begin{bmatrix}
            6 \\ 10 \\ 14
        \end{bmatrix}
    \]
    \item 
    \[ || x || = \sqrt{5}\]
    \item 
    \[ x^T = \begin{bmatrix} 
            0  & 
            1  & 
            2  
        \end{bmatrix}
    \]
    \item 
    \[
        Ax = \begin{bmatrix}
            0 & 0 & 0 \\
            1 & 3 & 1 \\
            2 & 2 & 6 \\
        \end{bmatrix}
    \]
    \item 
    \[
        x^T A x = 
        \begin{bmatrix}
            0 &  0 &  0 \\
            0 &  3 &  2 \\
            0 &  2 & 12 \\
        \end{bmatrix}
    \]
\end{enumerate}

% Section 1.2
\subsection{Matrix Algebra Rules}
\begin{enumerate}
    \item True
    \item True
    \item False
    \item True
    \item False
    \item True
    \item True
    \item True
    \item True
\end{enumerate}

% Section 1.3
\subsection{Matrix operations}
\begin{itemize}
    \item Givens
    \[
        B = \begin{bmatrix}
            1  & -1 & 0  \\ 
            -1 & 2  & -1 \\ 
            0  & -1 & 1   
        \end{bmatrix}
    \]
    \item Is B invertible? If so find $B^{-1}$
    \[
        det(B) = -(-1) (-1) + (1) (2-1) = -1 + 1 = 0
    \]
    B is not invertible

    \item Is B diagonalizable? If so, find its diagonalization
        \begin{itemize}
            \item Eigenvalues
            \[
                det(B - \lambda I) = 0 \longrightarrow
                \lambda_1 = 0, \lambda_2 = 2, \lambda_3 = 2
            \]
            \item Eigenvectors
            \[
                (B - 0I)v = 0 \longrightarrow v_1 = \begin{bmatrix}
                    1 \\ 1 \\ 1
                \end{bmatrix}
            \]
            \[
                (B - 2I)v = 0 \longrightarrow 
                v_2 = 
                \begin{bmatrix}
                    -1 \\ 1 \\ 0
                \end{bmatrix}, 
                v_3
                \begin{bmatrix}
                    -1 \\ 0 \\ 1
                \end{bmatrix}
            \]
            \item Calculate P
            \[ P =
                \begin{bmatrix}
                    v_1 & v_2 & v_3
                \end{bmatrix} =
                \begin{bmatrix}
                    1 & -1 & -1 \\
                    1 & 1  & 0  \\
                    1 & 0  & 1
                \end{bmatrix}
            \]
            \item Calculate D
            \[ D = 
                \begin{bmatrix}
                    \lambda_1 & 0 & 0 \\ 
                    0 & \lambda_1 & 0 \\ 
                    0 & 0 & \lambda_3
                \end{bmatrix} =
                \begin{bmatrix}
                    0 & 0 & 0 \\ 
                    0 & 2 & 0 \\ 
                    0 & 0 & 2
                \end{bmatrix}
            \]
            \item Diagonalization
            \[
                B = PDP^{-1}
            \]
        \end{itemize}
\end{itemize}

\section{Taking Chances: Probability Review}
\subsection{Basic Probability}
\begin{enumerate}
    \item Tickets should be \$5
    \[
        P(win) = 12/36 = 1/3
    \]
    \[
        P(loss) = 1 - P(win) = 2/3
    \]
    \[
        EV = 15(1/3) + 0(2/3) = 5
    \]
    \item P(B) = .95 - .4 = .55
    \item 
    \[
        P(A \cup B) = P(A) + P(B) - P(A)P(B)
    \]
    \[
        .95 = .4 + .6 P(B) \longrightarrow P(B) = .9167
    \]
\end{enumerate}

\subsection{Expectations and Variance}
\begin{enumerate}
    \item 
    \[
        P(H) = .5 (.3+.9) = .6
    \]
    \[
        E[X] = (3)(.6) = 1.8
    \]
    \item 
    \[
        Var(X) = 3(.6)(.4) = .72
    \]
\end{enumerate}

\subsection{A Variance Paradox?}
\begin{enumerate}
    \item There's no contradiction. When you sum i.i.d. random variables, you can sum their variance. However, when you add a random variable to itself, it's not i.i.d, instead you scale its values by $2x$. Therefore $Var(X + X) = Var(2X) = 4Var(X)$
\end{enumerate}

\section{Calculus Review }
\subsection{One-variable derivatives}
\begin{itemize}
    \item 
    \[
        f^{'}(x) = 6x-2
    \]
    \item 
    \[
        f^{'}(x) = 1 - 2x
    \]
    \item 
    \[
        f^{'}(x) = 1 - \frac{\exp(-x)}{1+\exp(-x)} = 1 - (1-p(x)) = p(x)
    \]
\end{itemize}

\subsection{Multi-variable derivative}
\begin{itemize}
    \item 
    \[
        \Delta f(x) = [x_1^2, exp(x_2)]
    \]
    \item 
    \[
        \Delta f(x) = exp(x_1, x_2 x_3), x_3 exp(x_1, x_2 x_3), x_2 exp(x_1, x_2 x_3)]
    \]
    \item 
    \[
        \Delta f(x) = a
    \]
    \item 
    \[\Delta f(x) = 2Ax =
        \begin{bmatrix}
             4x_1 - 2x_2 \\
            -2x_1 + 2x_2
        \end{bmatrix}
    \]
    \item 
    \[ 
        \Delta f(x) = x 
    \]
\end{itemize}

\section{Algorithms and Data Structures Review}
\begin{enumerate}
    \item 
    \[
        O(n log(n))
    \]
    \item 
    \[
        O(n)
    \]
    \item 
    \[
        O(log(n))
    \]
    \item 
    \[
        O(1)
    \]
    \item 
    \[
        O(nd)
    \]
    \item 
    \[
        O(d^2)
    \]
    \item 
    \[
        O(mnd)
    \]
\end{enumerate}
\end{document}